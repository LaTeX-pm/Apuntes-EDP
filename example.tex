\chapter{Introducción}

\section{}
Electroestática

$dim \vect{E} = 3$ 

Ecuaciones de Poisson
(insertar imagen)
Lo que se hace usualmente se tiene un material (el cuál en matempatica se llama dominio)

Otro tipo importante de problemas son los Problemas que dependen del tiempo.

Si ahora las cosas de eléctrica dependen del tiempo, entonces el rotor no es 0y todo el escenario cambia.

\begin{example}
\begin{equation}
    \partial_t^2 E_x - c^2 \nabla E_x = 1
\end{equation}


Es la ecuación de ondas.

Uno piensa en fijar $E(t=0)$ y $\partial_t T(t=0)$ y se calcula. Es un prototipo de ecuación para fenómenos que conservan energía.

Hay otra ecuación:
\begin{equation}
    -i \partial_t \psi = - \nabla \psi + V \psi
\end{equation}

La cuál es la Ec. de Scrödinger

Hay otra ecuación similar que es la ecuación del calor $u(x, T)$:
\begin{eqnarray}
    \vect{q} 
    &= 
    - {\nabla} 
    \cdot T\\
    \partial_t u &= \nabla \cdot \vect{q}
\end{eqnarray}

Donde $q$ es el flujo de calor

$\vect{u} (\vect{x}, t)$ velocidad elemento de fluido
\begin{equation}
    \partial_t \vect{u} + \vect{u} \cdot \nabla \vect{u} = - \nabla p + \mu \Delta \vect{u}
\end{equation}

Esto nos muestra que existe una jungla de ecuaciones

Recomendación de libro:

para ir ``a pie'' se recomienda Jeffrey Rauch PDE, es de Springer. Es un libro del acercamiento más clásico.

Mientras que ``a helicóptero'', un bueno acercamiento es Brezis. Para la parte de Espacio de Sobolev, se ocupará este libro.

Liev Loss Analysis (se sacarán algunas demostraciones de este libro).

Evans PDE. Es un libro que queda muy grande para el curso.

\end{example}

¿Cómo vamos a funcionar? Este es un curso bastante pesado. Así que se tiene una propuesta: ver menos materia pero verlas bien en profundidad. También se tenía planeado en tener 2 cátedras y dejar la última hora. Pero este miércoles Sí habrá clases.

Evaluaciones: La idea es trabajar durante el semestre. Un de las ideas que se le ocurre es hacer un apunte juntos. Otra idea es presentar un proyecto. O si es muy complicado, también existe un control (que será un tercio de la materia). Y finalmente un examen. Sí o sí el examen será oral obligatorio.